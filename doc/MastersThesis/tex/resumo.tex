
Fenômenos físicos são difíceis de modelar propriamente devido a sua natureza contínua. O paralelismo e nuances envolvidos eram um desafio
antes do transistor, e mesmo depois do computador digital esse problema continua insolúvel. No passado, algum formalismo foi trazido pelo
computador analógico de propósito geral (GPAC) por Shannon nos anos 1940. Infelizmente, essa base formal foi perdida com o tempo, e práticas
\textit{ad-hoc} tornaram-se comuns para simular o tempo contínuo. Neste trabalho, propomos uma linguagem de domínio específico (DSL) -- FACT e sua
evolução FFACT -- escrita
em Haskell que assemelha-se aos conceitos do GPAC. O principal objetivo é aproveitar de abstrações de mais alto nível, tanto da área da programação quanto
da matemática, para executar sistemas de equações diferenciais, que descrevem sistemas físicos matematicamente. Nós avaliamos performance and problemas de
domínio e os endereçamos propriamente. Melhorias futuras para a DSL também são exploradas e datalhadas.
