
Physical phenomena is difficult to properly model due to its continuous nature. Its paralellism and nuances were a challenge before the transistor, and even after the digital computer still is an unsolved issue. In the past, some formalism were brought with the General Purpose Analog Computer proposed by Shannon in the 1940s. Unfortunately, this formal foundation was lost in time, with \textit{ad-hoc} practices becoming mainstream to simulate continuous time. In this work, we propose a domain-specific language (DSL) written in Haskell that resembles GPAC's concepts. The main goal is to take advantage of high level abtractions to execute systems of differential equations, which describe physical problems mathematically. We evaluate performance and domain problems and address them accordingly. Future improvements for the DSL are also explored and detailed.
